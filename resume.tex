%-------------------------
% Resume in Latex
% Author : Daniel Curilla
% Template: https://www.overleaf.com/latex/templates/jakes-resume/syzfjbzwjncs
% License : MIT
%------------------------

\documentclass[letterpaper,11pt]{article}

\usepackage{latexsym}
\usepackage[empty]{fullpage}
\usepackage{titlesec}
\usepackage{marvosym}
\usepackage[usenames,dvipsnames]{color}
\usepackage{verbatim}
\usepackage{enumitem}
\usepackage[hidelinks]{hyperref}
\usepackage{fancyhdr}
\usepackage[english]{babel}
\usepackage{tabularx}
\usepackage{xcolor}
\input{glyphtounicode}



%----------FONT OPTIONS----------
% sans-serif
% \usepackage[sfdefault]{FiraSans}
% \usepackage[sfdefault]{roboto}
% \usepackage[sfdefault]{noto-sans}
% \usepackage[default]{sourcesanspro}

% serif
% \usepackage{CormorantGaramond}
% \usepackage{charter}


\pagestyle{fancy}
\fancyhf{} % clear all header and footer fields
\fancyfoot{}
\renewcommand{\headrulewidth}{0pt}
\renewcommand{\footrulewidth}{0pt}

% Adjust margins
\addtolength{\oddsidemargin}{-0.5in}
\addtolength{\evensidemargin}{-0.5in}
\addtolength{\textwidth}{1in}
\addtolength{\topmargin}{-.5in}
\addtolength{\textheight}{1.0in}

\urlstyle{same}

\raggedbottom
\raggedright
\setlength{\tabcolsep}{0in}

% Sections formatting
\titleformat{\section}{
  \vspace{-4pt}\scshape\raggedright\large
}{}{0em}{}[\color{black}\titlerule \vspace{-5pt}]

% Ensure that generate pdf is machine readable/ATS parsable
\pdfgentounicode=1

%-------------------------
% Custom commands
\newcommand{\resumeItem}[1]{
  \item\small{
    {#1 \vspace{-2pt}}
  }
}

\newcommand{\resumeSubheading}[4]{
  \vspace{-2pt}\item
    \begin{tabular*}{0.97\textwidth}[t]{l@{\extracolsep{\fill}}r}
      \textbf{#1} & #2 \\
      \textit{\small#3} & \textit{\small #4} \\
    \end{tabular*}\vspace{-7pt}
}

\newcommand{\resumeSubSubheading}[2]{
    \item
    \begin{tabular*}{0.97\textwidth}{l@{\extracolsep{\fill}}r}
      \textit{\small#1} & \textit{\small #2} \\
    \end{tabular*}\vspace{-7pt}
}

\newcommand{\resumeProjectHeading}[2]{
    \item
    \begin{tabular*}{0.97\textwidth}{l@{\extracolsep{\fill}}r}
      \small#1 & #2 \\
    \end{tabular*}\vspace{-7pt}
}

\renewcommand\labelitemii{$\vcenter{\hbox{\tiny$\bullet$}}$}
\renewcommand\labelitemiii{$\textendash$}

\newcommand{\resumeSubHeadingListStart}{\begin{itemize}[leftmargin=0.15in, label={}]}
\newcommand{\resumeSubHeadingListEnd}{\end{itemize}}
\newcommand{\resumeItemListStart}{\begin{itemize}}
\newcommand{\resumeItemListEnd}{\end{itemize}\vspace{-5pt}}

%-------------------------------------------
%%%%%%  RESUME STARTS HERE  %%%%%%%%%%%%%%%%%%%%%%%%%%%%


\begin{document}

\begin{center}
    \textbf{\Huge \scshape Daniel Curilla} \\ \vspace{1pt}
    \small 403-472-6945 $|$ \href{curilladaniel@gmail.com}{\color{blue}\underline{curilladaniel@gmail.com}} $|$ 
    \href{https://www.linkedin.com/in/daniel-curilla-252bb6149/}{\underline{\color{blue} linkedin}} $|$
    \href{https://github.com/djcurill}{\underline{\color{blue} github}}
\end{center}


%-----------EXPERIENCE-----------
\section{Experience}
  \resumeSubHeadingListStart

    \resumeSubheading
      {Sr. Data Scientist}{Oct. 2019 -- Present}
      {IBM}{Calgary, Alberta}
      \resumeItemListStart
        \resumeItem{Acted as the Senior Data Scientist for high-profile mining clients in the development of IBM’s mineral 
        prospecting algorithms, responsible for solution architecture, implementation and technical communication 
        with clients and internal stakeholders}
        \resumeItem{Led the research and developement of a 2D Convolutional Neural Network formulation capable of identifying 
        exploration drill targets using multi-channel geophysical inputs}
        \resumeItem{Built a generalized deep learning pipeline to allow simulataneous ingestion of disparate data 
        sources including 2D geophysics data and 3D drillhole data}
        \resumeItem{Applied proper software design patterns during development and enabled compatibility with 
        popular Deep Learning framework (i.e PyTorch Lightning and fast.ai) to improve computation time and 
        model performances}
        \resumeItem{Managed a team of 3 data scientists and organized client deliverables and development 
        tasks using GitHub issue tracking, milestones and project boards to better align focus and 
        direction of technical team}
        \resumeItem{Onboarded new data scientists by hosting live programming sessions that demonstrate proper 
        python programming practices and in-depth technical reviews of the codebase to scale the teams ability to 
        deliver better exploration targets for more clients}
        \resumeItem{Presented the IBM DeepMine solution at the Calgary Artificial Intelligence MeetUp to grow 
        IBM’s eminence within the data science community and to share the lessons learned when building 
        deep learning solutions for geoscientific applications}
      \resumeItemListEnd

    \resumeSubheading
      {Jr. Data Scientist}{Oct. 2017 -- Oct. 2019}
      {IBM}{Calgary, Alberta}
      \resumeItemListStart
        \resumeItem{Designed and developed a python codebase that utilizes 3D Convolutional Neural Networks to 
        identify mining exploration targets based on geological drillhole data}
        \resumeItem{Applied software best practices such as Object Oriented design, GitHub version control and proper 
        unit testing coverage to ensure quality of exploration targets delivered to clients}
        \resumeItem{Developed a novel dimensionality reduction technique derived from Natural Language Processing 
        embedding methods (word2vec) specific to geological data that improved model performance while 
        greatly reducing computational burden of 3D data structures}
        \resumeItem{Migrated deep learning software from TensorFlow to PyTorch leading to the following features/enhancements:}
        \begin{itemize}
          \item Elegant callback API to improve model performance and simplify debugging process
          \item Stochastic data augmentation techniques leading to new state of the art performances
          \item 3X speed up (measured by the relative increase of deliverables 
          capable of being provided to the client over a single week)
        \end{itemize}
      \resumeItemListEnd

  \resumeSubHeadingListEnd

%-----------EDUCATION-----------
\section{Education}
  \resumeSubHeadingListStart
    \resumeSubheading
      {University of Calgary}{Calgary, Alberta}
      {Dual BSc Geophysics and Mathematics}{2012 -- 2017}
  \resumeSubHeadingListEnd


%-----------PROJECTS-----------
\section{Projects}
    \resumeSubHeadingListStart
      \resumeProjectHeading
          {\textbf{Mini Games} $|$ \emph{Python, Software Design, Algorithms, Technical Writing}}{}
          \resumeItemListStart
            \resumeItem{A series of \href{https://github.com/djcurill/mini-games}{\color{blue} blog posts} in the form 
            of Jupyter Notebooks that rationalize my thought process when solving problems related software design, 
            algorithmic thinking and teaching programming fundamentals}
            \resumeItem{Project focus is to improve python programming skills and technical writing capabilities}
          \resumeItemListEnd

      \resumeProjectHeading
          {\textbf{Odin Project} $|$ \emph{JavaScript, NodeJs, HTML and CSS}}{}
          \resumeItemListStart
            \resumeItem{Working through the \href{https://www.theodinproject.com/}{\color{blue}Odin Project} to learn more
            about Full Stack Development and strenghten my software development skills for new languages and applications}
          \resumeItemListEnd
          
      \resumeProjectHeading
          {\textbf{LaTex Build of Resume} $|$ \emph{LaTex and Git Version Control}}{}
          \resumeItemListStart
            \resumeItem{Learned Latex to create this \href{https://github.com/djcurill/resume}{\color{blue}resume} and 
            applied Git version control to demonstrate code management capabilities}
          \resumeItemListEnd

      \resumeProjectHeading
          {\textbf{YYC DataCon (March 12th, 2021)} $|$ \emph{Public Speaking}}{}
          \resumeItemListStart
            \resumeItem{Sharing with the Data Science Community the lessons our team learned when applying Deep Learning 
            to geoscience application }
            \resumeItem{The presentation will cover the challenges associated with spatial data for machine learning, applying 
            proper software development best practices and utilizing productivity tools to scale up technical staff}
          \resumeItemListEnd
    \resumeSubHeadingListEnd



%
%-----------TECHNICAL SKILLS-----------
\section{Technical Skills}
 \begin{itemize}[leftmargin=0.15in, label={}]
    \small{\item{
     \textbf{Languages}{: Python, PostgreSQL, HTML, CSS} \\
     \textbf{Machine Learning Libraries}{: PyTorch, Keras, fast.ai, Scikit-Learn} \\
     \textbf{Developper Tools}{: Github, Bash, Anaconda, Regex} \\
    }}
 \end{itemize}


%-------------------------------------------
\end{document}